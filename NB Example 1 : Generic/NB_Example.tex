\documentclass[]{article}
\usepackage{lmodern}
\usepackage{amssymb,amsmath}
\usepackage{ifxetex,ifluatex}
\usepackage{fixltx2e} % provides \textsubscript
\ifnum 0\ifxetex 1\fi\ifluatex 1\fi=0 % if pdftex
  \usepackage[T1]{fontenc}
  \usepackage[utf8]{inputenc}
\else % if luatex or xelatex
  \ifxetex
    \usepackage{mathspec}
  \else
    \usepackage{fontspec}
  \fi
  \defaultfontfeatures{Ligatures=TeX,Scale=MatchLowercase}
\fi
% use upquote if available, for straight quotes in verbatim environments
\IfFileExists{upquote.sty}{\usepackage{upquote}}{}
% use microtype if available
\IfFileExists{microtype.sty}{%
\usepackage{microtype}
\UseMicrotypeSet[protrusion]{basicmath} % disable protrusion for tt fonts
}{}
\usepackage[margin=1in]{geometry}
\usepackage{hyperref}
\hypersetup{unicode=true,
            pdftitle={Naive Bayes - An example using R language},
            pdfborder={0 0 0},
            breaklinks=true}
\urlstyle{same}  % don't use monospace font for urls
\usepackage{color}
\usepackage{fancyvrb}
\newcommand{\VerbBar}{|}
\newcommand{\VERB}{\Verb[commandchars=\\\{\}]}
\DefineVerbatimEnvironment{Highlighting}{Verbatim}{commandchars=\\\{\}}
% Add ',fontsize=\small' for more characters per line
\usepackage{framed}
\definecolor{shadecolor}{RGB}{248,248,248}
\newenvironment{Shaded}{\begin{snugshade}}{\end{snugshade}}
\newcommand{\AlertTok}[1]{\textcolor[rgb]{0.94,0.16,0.16}{#1}}
\newcommand{\AnnotationTok}[1]{\textcolor[rgb]{0.56,0.35,0.01}{\textbf{\textit{#1}}}}
\newcommand{\AttributeTok}[1]{\textcolor[rgb]{0.77,0.63,0.00}{#1}}
\newcommand{\BaseNTok}[1]{\textcolor[rgb]{0.00,0.00,0.81}{#1}}
\newcommand{\BuiltInTok}[1]{#1}
\newcommand{\CharTok}[1]{\textcolor[rgb]{0.31,0.60,0.02}{#1}}
\newcommand{\CommentTok}[1]{\textcolor[rgb]{0.56,0.35,0.01}{\textit{#1}}}
\newcommand{\CommentVarTok}[1]{\textcolor[rgb]{0.56,0.35,0.01}{\textbf{\textit{#1}}}}
\newcommand{\ConstantTok}[1]{\textcolor[rgb]{0.00,0.00,0.00}{#1}}
\newcommand{\ControlFlowTok}[1]{\textcolor[rgb]{0.13,0.29,0.53}{\textbf{#1}}}
\newcommand{\DataTypeTok}[1]{\textcolor[rgb]{0.13,0.29,0.53}{#1}}
\newcommand{\DecValTok}[1]{\textcolor[rgb]{0.00,0.00,0.81}{#1}}
\newcommand{\DocumentationTok}[1]{\textcolor[rgb]{0.56,0.35,0.01}{\textbf{\textit{#1}}}}
\newcommand{\ErrorTok}[1]{\textcolor[rgb]{0.64,0.00,0.00}{\textbf{#1}}}
\newcommand{\ExtensionTok}[1]{#1}
\newcommand{\FloatTok}[1]{\textcolor[rgb]{0.00,0.00,0.81}{#1}}
\newcommand{\FunctionTok}[1]{\textcolor[rgb]{0.00,0.00,0.00}{#1}}
\newcommand{\ImportTok}[1]{#1}
\newcommand{\InformationTok}[1]{\textcolor[rgb]{0.56,0.35,0.01}{\textbf{\textit{#1}}}}
\newcommand{\KeywordTok}[1]{\textcolor[rgb]{0.13,0.29,0.53}{\textbf{#1}}}
\newcommand{\NormalTok}[1]{#1}
\newcommand{\OperatorTok}[1]{\textcolor[rgb]{0.81,0.36,0.00}{\textbf{#1}}}
\newcommand{\OtherTok}[1]{\textcolor[rgb]{0.56,0.35,0.01}{#1}}
\newcommand{\PreprocessorTok}[1]{\textcolor[rgb]{0.56,0.35,0.01}{\textit{#1}}}
\newcommand{\RegionMarkerTok}[1]{#1}
\newcommand{\SpecialCharTok}[1]{\textcolor[rgb]{0.00,0.00,0.00}{#1}}
\newcommand{\SpecialStringTok}[1]{\textcolor[rgb]{0.31,0.60,0.02}{#1}}
\newcommand{\StringTok}[1]{\textcolor[rgb]{0.31,0.60,0.02}{#1}}
\newcommand{\VariableTok}[1]{\textcolor[rgb]{0.00,0.00,0.00}{#1}}
\newcommand{\VerbatimStringTok}[1]{\textcolor[rgb]{0.31,0.60,0.02}{#1}}
\newcommand{\WarningTok}[1]{\textcolor[rgb]{0.56,0.35,0.01}{\textbf{\textit{#1}}}}
\usepackage{graphicx,grffile}
\makeatletter
\def\maxwidth{\ifdim\Gin@nat@width>\linewidth\linewidth\else\Gin@nat@width\fi}
\def\maxheight{\ifdim\Gin@nat@height>\textheight\textheight\else\Gin@nat@height\fi}
\makeatother
% Scale images if necessary, so that they will not overflow the page
% margins by default, and it is still possible to overwrite the defaults
% using explicit options in \includegraphics[width, height, ...]{}
\setkeys{Gin}{width=\maxwidth,height=\maxheight,keepaspectratio}
\IfFileExists{parskip.sty}{%
\usepackage{parskip}
}{% else
\setlength{\parindent}{0pt}
\setlength{\parskip}{6pt plus 2pt minus 1pt}
}
\setlength{\emergencystretch}{3em}  % prevent overfull lines
\providecommand{\tightlist}{%
  \setlength{\itemsep}{0pt}\setlength{\parskip}{0pt}}
\setcounter{secnumdepth}{0}
% Redefines (sub)paragraphs to behave more like sections
\ifx\paragraph\undefined\else
\let\oldparagraph\paragraph
\renewcommand{\paragraph}[1]{\oldparagraph{#1}\mbox{}}
\fi
\ifx\subparagraph\undefined\else
\let\oldsubparagraph\subparagraph
\renewcommand{\subparagraph}[1]{\oldsubparagraph{#1}\mbox{}}
\fi

%%% Use protect on footnotes to avoid problems with footnotes in titles
\let\rmarkdownfootnote\footnote%
\def\footnote{\protect\rmarkdownfootnote}

%%% Change title format to be more compact
\usepackage{titling}

% Create subtitle command for use in maketitle
\providecommand{\subtitle}[1]{
  \posttitle{
    \begin{center}\large#1\end{center}
    }
}

\setlength{\droptitle}{-2em}

  \title{Naive Bayes - An example using R language}
    \pretitle{\vspace{\droptitle}\centering\huge}
  \posttitle{\par}
    \author{}
    \preauthor{}\postauthor{}
    \date{}
    \predate{}\postdate{}
  

\begin{document}
\maketitle

The main purpose of this document is to educate the reader with the
practical aspect of the application of the Naive Bayes (NB) algorithm.
In this direction, here we provide the necessary code snippet to show
the reader how to apply the NB algorithm in a data classification
problem. We use a publicly available dataset and R language for this
demonstration. All the important steps of the implementation of the
algorithm are explined below.

\textbf{The dataset:} In order to implement the above code, we use the
well-known Iris dataset (included with R). It consists of four features
(measurements), namely sepal length, sepal width, petal length, and
petal width for 150 flowers. The dataset contains information about three
types of iris plants: Setosa, Versicolor and Virginica. In the R
package, likelihoods in the Naive Bayes formula for numerical features
(e.g.~measurements) are calculated using probability distributions.

\textbf{Machine learning (ML) task:} In this application, we aim to
construct an ML model using NB algorithm to identify (classify) the
specie of an Irish flower, based on the values of the four features (
sepal length, sepal width, petal length, and petal width ) of a given
Irish flower. So, we're tackling a classification task here.

\begin{Shaded}
\begin{Highlighting}[]
\KeywordTok{data}\NormalTok{(iris) }\CommentTok{# Attach the Iris dataset to the R environment}
\NormalTok{mydata <-}\StringTok{ }\NormalTok{iris }\CommentTok{# Let's rename the dataset as mydata}
\KeywordTok{dim}\NormalTok{(mydata) }\CommentTok{# check dimensions of mydata}
\end{Highlighting}
\end{Shaded}

\begin{verbatim}
## [1] 150   5
\end{verbatim}

\begin{Shaded}
\begin{Highlighting}[]
\KeywordTok{sapply}\NormalTok{(mydata, class) }\CommentTok{# check the data types of each feature}
\end{Highlighting}
\end{Shaded}

\begin{verbatim}
## Sepal.Length  Sepal.Width Petal.Length  Petal.Width      Species 
##    "numeric"    "numeric"    "numeric"    "numeric"     "factor"
\end{verbatim}

\begin{Shaded}
\begin{Highlighting}[]
\KeywordTok{levels}\NormalTok{(mydata}\OperatorTok{$}\NormalTok{Species) }\CommentTok{# check different levels (values) for each class}
\end{Highlighting}
\end{Shaded}

\begin{verbatim}
## [1] "setosa"     "versicolor" "virginica"
\end{verbatim}

\begin{Shaded}
\begin{Highlighting}[]
\KeywordTok{head}\NormalTok{(mydata) }\CommentTok{# have a look at top data points in mydata}
\end{Highlighting}
\end{Shaded}

\begin{verbatim}
##   Sepal.Length Sepal.Width Petal.Length Petal.Width Species
## 1          5.1         3.5          1.4         0.2  setosa
## 2          4.9         3.0          1.4         0.2  setosa
## 3          4.7         3.2          1.3         0.2  setosa
## 4          4.6         3.1          1.5         0.2  setosa
## 5          5.0         3.6          1.4         0.2  setosa
## 6          5.4         3.9          1.7         0.4  setosa
\end{verbatim}

\begin{Shaded}
\begin{Highlighting}[]
\KeywordTok{summary}\NormalTok{(mydata) }\CommentTok{# some descriptive statitics of the data and a summary of class distributions}
\end{Highlighting}
\end{Shaded}

\begin{verbatim}
##   Sepal.Length    Sepal.Width     Petal.Length    Petal.Width   
##  Min.   :4.300   Min.   :2.000   Min.   :1.000   Min.   :0.100  
##  1st Qu.:5.100   1st Qu.:2.800   1st Qu.:1.600   1st Qu.:0.300  
##  Median :5.800   Median :3.000   Median :4.350   Median :1.300  
##  Mean   :5.843   Mean   :3.057   Mean   :3.758   Mean   :1.199  
##  3rd Qu.:6.400   3rd Qu.:3.300   3rd Qu.:5.100   3rd Qu.:1.800  
##  Max.   :7.900   Max.   :4.400   Max.   :6.900   Max.   :2.500  
##        Species  
##  setosa    :50  
##  versicolor:50  
##  virginica :50  
##                 
##                 
## 
\end{verbatim}

\textbf{Creating training and validation datasets:} We're going to
follow the convention of 80/20 samples ratio to partition the dataset to
the training and validation sets. We use the createDataPartition
function from the caret package for this purpose.

\begin{Shaded}
\begin{Highlighting}[]
\CommentTok{#install.packages("caret") #If the caret package is not installed on your system, uncomment this line to install it first}
\KeywordTok{library}\NormalTok{(caret) }\CommentTok{#Loading the library}
\NormalTok{tr_index <-}\StringTok{ }\KeywordTok{createDataPartition}\NormalTok{(mydata}\OperatorTok{$}\NormalTok{Species, }\DataTypeTok{p=}\FloatTok{0.80}\NormalTok{, }\DataTypeTok{list=}\OtherTok{FALSE}\NormalTok{) }\CommentTok{# List of 80% of the rows}
\NormalTok{trainSet <-}\StringTok{ }\NormalTok{mydata[tr_index,] }\CommentTok{# select 80% of the data for the trainSet}
\NormalTok{testSet <-}\StringTok{ }\NormalTok{mydata[}\OperatorTok{-}\NormalTok{tr_index,] }\CommentTok{# Select the remaining 20% of data for testSet}
\end{Highlighting}
\end{Shaded}

\textbf{Building a NB classifier:} Now we will train our NB classifier
using the above trainSet. For this purpose, we will utilize e1071
package in R. Note that following lines of code will fit the NB model to
the datset and will also display the model details. In this case, the
priori probabilities for all classes are the same.

\begin{Shaded}
\begin{Highlighting}[]
\CommentTok{#install.packages("e1071") #If the e1071 package is not installed on your system, uncomment this line to install it first}
\KeywordTok{library}\NormalTok{(e1071)}
\NormalTok{NBclassfier=}\KeywordTok{naiveBayes}\NormalTok{(Species}\OperatorTok{~}\NormalTok{., }\DataTypeTok{data=}\NormalTok{trainSet) }\CommentTok{# Once you call this line, R fits the NB model using trainSet}
\KeywordTok{print}\NormalTok{(NBclassfier) }\CommentTok{# Check the newly fitted model to see if everything is OK.}
\end{Highlighting}
\end{Shaded}

\begin{verbatim}
## 
## Naive Bayes Classifier for Discrete Predictors
## 
## Call:
## naiveBayes.default(x = X, y = Y, laplace = laplace)
## 
## A-priori probabilities:
## Y
##     setosa versicolor  virginica 
##  0.3333333  0.3333333  0.3333333 
## 
## Conditional probabilities:
##             Sepal.Length
## Y              [,1]      [,2]
##   setosa     4.9800 0.3631769
##   versicolor 5.8900 0.5042384
##   virginica  6.6375 0.6019999
## 
##             Sepal.Width
## Y              [,1]      [,2]
##   setosa     3.4200 0.3943186
##   versicolor 2.7800 0.3022989
##   virginica  3.0025 0.3158119
## 
##             Petal.Length
## Y              [,1]      [,2]
##   setosa     1.4475 0.1839558
##   versicolor 4.2650 0.4907085
##   virginica  5.5825 0.5439115
## 
##             Petal.Width
## Y              [,1]       [,2]
##   setosa     0.2350 0.09753369
##   versicolor 1.3250 0.20223114
##   virginica  2.0825 0.24166888
\end{verbatim}

\textbf{Make predictions:} Now let's apply the above model to assign
labels for test cases in testSet. Then we create the confusion matrix, a
table that is often used to describe the performance of a classifier.

\begin{Shaded}
\begin{Highlighting}[]
\NormalTok{  testPrediction=}\KeywordTok{predict}\NormalTok{(NBclassfier, }\DataTypeTok{newdata=}\NormalTok{testSet, }\DataTypeTok{type=}\StringTok{"class"}\NormalTok{) }\CommentTok{# Assign labels for each test case}
  \KeywordTok{confusionMatrix}\NormalTok{(testPrediction, testSet}\OperatorTok{$}\NormalTok{Species) }\CommentTok{# Print confusion matrix }
\end{Highlighting}
\end{Shaded}

\begin{verbatim}
## Confusion Matrix and Statistics
## 
##             Reference
## Prediction   setosa versicolor virginica
##   setosa         10          0         0
##   versicolor      0         10         4
##   virginica       0          0         6
## 
## Overall Statistics
##                                           
##                Accuracy : 0.8667          
##                  95% CI : (0.6928, 0.9624)
##     No Information Rate : 0.3333          
##     P-Value [Acc > NIR] : 2.296e-09       
##                                           
##                   Kappa : 0.8             
##                                           
##  Mcnemar's Test P-Value : NA              
## 
## Statistics by Class:
## 
##                      Class: setosa Class: versicolor Class: virginica
## Sensitivity                 1.0000            1.0000           0.6000
## Specificity                 1.0000            0.8000           1.0000
## Pos Pred Value              1.0000            0.7143           1.0000
## Neg Pred Value              1.0000            1.0000           0.8333
## Prevalence                  0.3333            0.3333           0.3333
## Detection Rate              0.3333            0.3333           0.2000
## Detection Prevalence        0.3333            0.4667           0.2000
## Balanced Accuracy           1.0000            0.9000           0.8000
\end{verbatim}


\end{document}
