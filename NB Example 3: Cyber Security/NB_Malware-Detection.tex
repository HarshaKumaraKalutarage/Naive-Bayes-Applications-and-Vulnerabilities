\documentclass[]{article}
\usepackage{lmodern}
\usepackage{amssymb,amsmath}
\usepackage{ifxetex,ifluatex}
\usepackage{fixltx2e} % provides \textsubscript
\ifnum 0\ifxetex 1\fi\ifluatex 1\fi=0 % if pdftex
  \usepackage[T1]{fontenc}
  \usepackage[utf8]{inputenc}
\else % if luatex or xelatex
  \ifxetex
    \usepackage{mathspec}
  \else
    \usepackage{fontspec}
  \fi
  \defaultfontfeatures{Ligatures=TeX,Scale=MatchLowercase}
\fi
% use upquote if available, for straight quotes in verbatim environments
\IfFileExists{upquote.sty}{\usepackage{upquote}}{}
% use microtype if available
\IfFileExists{microtype.sty}{%
\usepackage{microtype}
\UseMicrotypeSet[protrusion]{basicmath} % disable protrusion for tt fonts
}{}
\usepackage[margin=1in]{geometry}
\usepackage{hyperref}
\hypersetup{unicode=true,
            pdftitle={Malware Classification},
            pdfborder={0 0 0},
            breaklinks=true}
\urlstyle{same}  % don't use monospace font for urls
\usepackage{color}
\usepackage{fancyvrb}
\newcommand{\VerbBar}{|}
\newcommand{\VERB}{\Verb[commandchars=\\\{\}]}
\DefineVerbatimEnvironment{Highlighting}{Verbatim}{commandchars=\\\{\}}
% Add ',fontsize=\small' for more characters per line
\usepackage{framed}
\definecolor{shadecolor}{RGB}{248,248,248}
\newenvironment{Shaded}{\begin{snugshade}}{\end{snugshade}}
\newcommand{\AlertTok}[1]{\textcolor[rgb]{0.94,0.16,0.16}{#1}}
\newcommand{\AnnotationTok}[1]{\textcolor[rgb]{0.56,0.35,0.01}{\textbf{\textit{#1}}}}
\newcommand{\AttributeTok}[1]{\textcolor[rgb]{0.77,0.63,0.00}{#1}}
\newcommand{\BaseNTok}[1]{\textcolor[rgb]{0.00,0.00,0.81}{#1}}
\newcommand{\BuiltInTok}[1]{#1}
\newcommand{\CharTok}[1]{\textcolor[rgb]{0.31,0.60,0.02}{#1}}
\newcommand{\CommentTok}[1]{\textcolor[rgb]{0.56,0.35,0.01}{\textit{#1}}}
\newcommand{\CommentVarTok}[1]{\textcolor[rgb]{0.56,0.35,0.01}{\textbf{\textit{#1}}}}
\newcommand{\ConstantTok}[1]{\textcolor[rgb]{0.00,0.00,0.00}{#1}}
\newcommand{\ControlFlowTok}[1]{\textcolor[rgb]{0.13,0.29,0.53}{\textbf{#1}}}
\newcommand{\DataTypeTok}[1]{\textcolor[rgb]{0.13,0.29,0.53}{#1}}
\newcommand{\DecValTok}[1]{\textcolor[rgb]{0.00,0.00,0.81}{#1}}
\newcommand{\DocumentationTok}[1]{\textcolor[rgb]{0.56,0.35,0.01}{\textbf{\textit{#1}}}}
\newcommand{\ErrorTok}[1]{\textcolor[rgb]{0.64,0.00,0.00}{\textbf{#1}}}
\newcommand{\ExtensionTok}[1]{#1}
\newcommand{\FloatTok}[1]{\textcolor[rgb]{0.00,0.00,0.81}{#1}}
\newcommand{\FunctionTok}[1]{\textcolor[rgb]{0.00,0.00,0.00}{#1}}
\newcommand{\ImportTok}[1]{#1}
\newcommand{\InformationTok}[1]{\textcolor[rgb]{0.56,0.35,0.01}{\textbf{\textit{#1}}}}
\newcommand{\KeywordTok}[1]{\textcolor[rgb]{0.13,0.29,0.53}{\textbf{#1}}}
\newcommand{\NormalTok}[1]{#1}
\newcommand{\OperatorTok}[1]{\textcolor[rgb]{0.81,0.36,0.00}{\textbf{#1}}}
\newcommand{\OtherTok}[1]{\textcolor[rgb]{0.56,0.35,0.01}{#1}}
\newcommand{\PreprocessorTok}[1]{\textcolor[rgb]{0.56,0.35,0.01}{\textit{#1}}}
\newcommand{\RegionMarkerTok}[1]{#1}
\newcommand{\SpecialCharTok}[1]{\textcolor[rgb]{0.00,0.00,0.00}{#1}}
\newcommand{\SpecialStringTok}[1]{\textcolor[rgb]{0.31,0.60,0.02}{#1}}
\newcommand{\StringTok}[1]{\textcolor[rgb]{0.31,0.60,0.02}{#1}}
\newcommand{\VariableTok}[1]{\textcolor[rgb]{0.00,0.00,0.00}{#1}}
\newcommand{\VerbatimStringTok}[1]{\textcolor[rgb]{0.31,0.60,0.02}{#1}}
\newcommand{\WarningTok}[1]{\textcolor[rgb]{0.56,0.35,0.01}{\textbf{\textit{#1}}}}
\usepackage{graphicx,grffile}
\makeatletter
\def\maxwidth{\ifdim\Gin@nat@width>\linewidth\linewidth\else\Gin@nat@width\fi}
\def\maxheight{\ifdim\Gin@nat@height>\textheight\textheight\else\Gin@nat@height\fi}
\makeatother
% Scale images if necessary, so that they will not overflow the page
% margins by default, and it is still possible to overwrite the defaults
% using explicit options in \includegraphics[width, height, ...]{}
\setkeys{Gin}{width=\maxwidth,height=\maxheight,keepaspectratio}
\IfFileExists{parskip.sty}{%
\usepackage{parskip}
}{% else
\setlength{\parindent}{0pt}
\setlength{\parskip}{6pt plus 2pt minus 1pt}
}
\setlength{\emergencystretch}{3em}  % prevent overfull lines
\providecommand{\tightlist}{%
  \setlength{\itemsep}{0pt}\setlength{\parskip}{0pt}}
\setcounter{secnumdepth}{0}
% Redefines (sub)paragraphs to behave more like sections
\ifx\paragraph\undefined\else
\let\oldparagraph\paragraph
\renewcommand{\paragraph}[1]{\oldparagraph{#1}\mbox{}}
\fi
\ifx\subparagraph\undefined\else
\let\oldsubparagraph\subparagraph
\renewcommand{\subparagraph}[1]{\oldsubparagraph{#1}\mbox{}}
\fi

%%% Use protect on footnotes to avoid problems with footnotes in titles
\let\rmarkdownfootnote\footnote%
\def\footnote{\protect\rmarkdownfootnote}

%%% Change title format to be more compact
\usepackage{titling}

% Create subtitle command for use in maketitle
\providecommand{\subtitle}[1]{
  \posttitle{
    \begin{center}\large#1\end{center}
    }
}

\setlength{\droptitle}{-2em}

  \title{Malware Classification}
    \pretitle{\vspace{\droptitle}\centering\huge}
  \posttitle{\par}
    \author{}
    \preauthor{}\postauthor{}
    \date{}
    \predate{}\postdate{}
  

\begin{document}
\maketitle

This code snippet shows you how to train a Naive Bayes (NB) algorithm
for malware classification. For this purpose, we can use different
features to train our algorithm. This includes features that are
obtained using both static code analysis and/or analysis of the dynamic
behavior of the malware.

\textbf{The dataset:} We will use the dataset published at
\url{https://data.mendeley.com/datasets/w2w8gjsgnt/1}
(\url{http://dx.doi.org/10.17632/w2w8gjsgnt.1\#file-6806d890-e13f-4644-abc2-630cca78216f}).
This data set contains a total of 1944 features that are obtained from
the static and dynamic analysis of \textasciitilde{} 19400 malware
samples including malware samples from APT attacks. We pre-processed the
dataset by removing the first three columns and the last seven columns.
Instead, we added a label column at the end of the dataset.

\textbf{Machine learning (ML) task:} We want to create an NB model to
identify (classify) the type of malware, so we are solving a
classification problem here.

\begin{Shaded}
\begin{Highlighting}[]
\NormalTok{myDataOrg <-}\StringTok{ }\KeywordTok{read.csv}\NormalTok{(}\StringTok{"malware.csv"}\NormalTok{, }\DataTypeTok{header=}\NormalTok{T)}
\KeywordTok{dim}\NormalTok{(myDataOrg) }\CommentTok{# check dimensions of myData}
\end{Highlighting}
\end{Shaded}

\begin{verbatim}
## [1] 19457  1935
\end{verbatim}

\begin{Shaded}
\begin{Highlighting}[]
\KeywordTok{levels}\NormalTok{(myDataOrg}\OperatorTok{$}\NormalTok{label) }\CommentTok{# check different levels (values) for each class}
\end{Highlighting}
\end{Shaded}

\begin{verbatim}
## [1] "Backdoor"  "OtherType" "Rootkit"   "Spyware"   "Trojan"    "Unknown"  
## [7] "Worm"
\end{verbatim}

\textbf{Check the balance of the dataset} : Class imbalance is very
common issue in Cyber Security dataset, and it's pretty common to expect
ratio like 1:100000 between classes (e.g.~attack:normal) due to the
scarcity of attack data.

Class imbalance is a very common problem in cyber security datasets, and
it is quite common that a 1:100000 ratio between classes (e.g.~attack:
normal) due to the scarcity of attack data. If the class imbalance
occurred then it can be affected on model performance. Let's look at
ratio between classes in our dataset.

\begin{Shaded}
\begin{Highlighting}[]
\KeywordTok{print}\NormalTok{(}\KeywordTok{table}\NormalTok{(myDataOrg}\OperatorTok{$}\NormalTok{label))}
\end{Highlighting}
\end{Shaded}

\begin{verbatim}
## 
##  Backdoor OtherType   Rootkit   Spyware    Trojan   Unknown      Worm 
##        53      1275       789       709     11034      3957      1640
\end{verbatim}

As you can see, our dataset is a hugely imbalance, especially Backdoor:
Trojan. A mix of oversampling and undersampling methods could be
utilised to balance the dataset, e.g., by increasing the size of
Backdoor class and reducing the size of Trojan class. However, this can
be resulted information lost in the larger class. Therefore, we will
split the dataset into two sets and train two NB models separately
(Ensamble technique).

\textbf{Note}: It should be noted that NB woud not be the best option
for this type of dataset, instead we recommend to try out some ensamble
techniques. However, we will train NB model in this way as our goal in
this post to show you how to train the NB model for this dataset
regardless of the accuarcy.

\begin{Shaded}
\begin{Highlighting}[]
\NormalTok{myDataSet1<-}\KeywordTok{rbind}\NormalTok{(}\KeywordTok{subset}\NormalTok{(myDataOrg, label }\OperatorTok{==}\StringTok{ "Backdoor"}\NormalTok{),}\KeywordTok{subset}\NormalTok{(myDataOrg, label }\OperatorTok{==}\StringTok{ "OtherType"}\NormalTok{),}\KeywordTok{subset}\NormalTok{(myDataOrg, label }\OperatorTok{==}\StringTok{ "Rootkit"}\NormalTok{),}\KeywordTok{subset}\NormalTok{(myDataOrg, label }\OperatorTok{==}\StringTok{ "Spyware"}\NormalTok{))}

\NormalTok{myDataSet2<-}\KeywordTok{rbind}\NormalTok{(}\KeywordTok{subset}\NormalTok{(myDataOrg, label }\OperatorTok{==}\StringTok{ "Trojan"}\NormalTok{),}\KeywordTok{subset}\NormalTok{(myDataOrg, label }\OperatorTok{==}\StringTok{ "Unknown"}\NormalTok{),}\KeywordTok{subset}\NormalTok{(myDataOrg, label }\OperatorTok{==}\StringTok{ "Worm"}\NormalTok{))}
\end{Highlighting}
\end{Shaded}

We split the original datset into two subsets. Let's continue with
myDataSet1. You can follow the same approach for myDataSet2. The
following code makes class sizes equal in myDataSet1.

\begin{Shaded}
\begin{Highlighting}[]
\NormalTok{sample.df <-}\StringTok{ }\ControlFlowTok{function}\NormalTok{(df, n) df[}\KeywordTok{sample}\NormalTok{(}\KeywordTok{nrow}\NormalTok{(df), n,}\DataTypeTok{replace =}\NormalTok{ T), , drop =}\StringTok{ }\NormalTok{F]}

\NormalTok{classSize<-}\DecValTok{500} \CommentTok{# sample from each class size of 500 records}

\NormalTok{myData<-}\KeywordTok{rbind}\NormalTok{(}\KeywordTok{sample.df}\NormalTok{(}\KeywordTok{subset}\NormalTok{(myDataSet1, label }\OperatorTok{==}\StringTok{ "Backdoor"}\NormalTok{), classSize),}\KeywordTok{sample.df}\NormalTok{(}\KeywordTok{subset}\NormalTok{(myDataSet1, label }\OperatorTok{==}\StringTok{ "OtherType"}\NormalTok{), classSize),}\KeywordTok{sample.df}\NormalTok{(}\KeywordTok{subset}\NormalTok{(myDataSet1, label }\OperatorTok{==}\StringTok{ "Rootkit"}\NormalTok{), classSize),}\KeywordTok{sample.df}\NormalTok{(}\KeywordTok{subset}\NormalTok{(myDataSet1, label }\OperatorTok{==}\StringTok{ "Spyware"}\NormalTok{), classSize))}
\end{Highlighting}
\end{Shaded}

Since we want to represent the presence or absence of a certain
static/dynamic feature in the malware, we code our dataset as follows.

\begin{Shaded}
\begin{Highlighting}[]
\NormalTok{convert_counts <-}\StringTok{ }\ControlFlowTok{function}\NormalTok{(x) \{}
\NormalTok{  x <-}\StringTok{ }\KeywordTok{ifelse}\NormalTok{(x }\OperatorTok{>}\StringTok{ }\DecValTok{0}\NormalTok{, }\StringTok{"Y"}\NormalTok{, }\StringTok{"N"}\NormalTok{)}
\NormalTok{\}}
\NormalTok{myData <-}\StringTok{ }\KeywordTok{data.frame}\NormalTok{(}\KeywordTok{apply}\NormalTok{(myData[,}\DecValTok{1}\OperatorTok{:}\DecValTok{1934}\NormalTok{], }\DataTypeTok{MARGIN =} \DecValTok{2}\NormalTok{,convert_counts),myData[}\DecValTok{1935}\NormalTok{] )}
\end{Highlighting}
\end{Shaded}

\textbf{Creating training and validation datasets:} We're going to
follow the convention of 80/20 samples ratio to partition the dataset to
the training and validation sets. We use the createDataPartition
function from the caret package for this purpose.

\begin{Shaded}
\begin{Highlighting}[]
\CommentTok{#install.packages("caret") #If the caret package is not installed on your system, uncomment this line to install it first}
\KeywordTok{set.seed}\NormalTok{(}\DecValTok{1234}\NormalTok{)}
\KeywordTok{library}\NormalTok{(caret) }\CommentTok{#Loading the library}
\NormalTok{tr_index <-}\StringTok{ }\KeywordTok{createDataPartition}\NormalTok{(myData}\OperatorTok{$}\NormalTok{label, }\DataTypeTok{p=}\FloatTok{0.80}\NormalTok{, }\DataTypeTok{list=}\OtherTok{FALSE}\NormalTok{) }\CommentTok{# List of 80% of the rows}
\end{Highlighting}
\end{Shaded}

\begin{verbatim}
## Warning in createDataPartition(myData$label, p = 0.8, list = FALSE): Some
## classes have no records ( Trojan, Unknown, Worm ) and these will be ignored
\end{verbatim}

\begin{Shaded}
\begin{Highlighting}[]
\NormalTok{trainSet <-}\StringTok{ }\NormalTok{myData[tr_index,] }\CommentTok{# select 80% of the data for the trainSet}
\NormalTok{testSet <-}\StringTok{ }\NormalTok{myData[}\OperatorTok{-}\NormalTok{tr_index,] }\CommentTok{# Select the remaining 20% of data for testSet}
\end{Highlighting}
\end{Shaded}

\textbf{Building a NB classifier:} Now we will train our NB classifier
using the above trainSet. For this purpose, we will utilize e1071
package in R. Note that the priori probabilities can be computed using
the following lines of code. In this case, the priori probabilities for
all classes are the same.

\begin{Shaded}
\begin{Highlighting}[]
\CommentTok{#install.packages("e1071") #If the e1071 package is not installed on your system, uncomment this line to install it first}
\KeywordTok{library}\NormalTok{(e1071)}
\NormalTok{NBclassfier <-}\StringTok{ }\KeywordTok{naiveBayes}\NormalTok{(trainSet[,}\DecValTok{1}\OperatorTok{:}\DecValTok{1934}\NormalTok{], trainSet}\OperatorTok{$}\NormalTok{label) }\CommentTok{# train the model}
\end{Highlighting}
\end{Shaded}

\textbf{Make predictions:} Now let's apply the above model to assign
labels for test cases in testSet. Then we create the confusion matrix, a
table that is often used to describe the performance of a classifier.

\begin{Shaded}
\begin{Highlighting}[]
\NormalTok{  testPrediction <-}\StringTok{ }\KeywordTok{predict}\NormalTok{(NBclassfier, testSet[,}\DecValTok{1}\OperatorTok{:}\DecValTok{1934}\NormalTok{]) }\CommentTok{# predict labels for test cases}
  \KeywordTok{confusionMatrix}\NormalTok{(testPrediction, testSet}\OperatorTok{$}\NormalTok{label) }\CommentTok{# Print confusion matrix }
\end{Highlighting}
\end{Shaded}

\begin{verbatim}
## Confusion Matrix and Statistics
## 
##            Reference
## Prediction  Backdoor OtherType Rootkit Spyware Trojan Unknown Worm
##   Backdoor        91        13      16      15      0       0    0
##   OtherType        0        47       7      17      0       0    0
##   Rootkit          4        16      62      16      0       0    0
##   Spyware          5        24      15      52      0       0    0
##   Trojan           0         0       0       0      0       0    0
##   Unknown          0         0       0       0      0       0    0
##   Worm             0         0       0       0      0       0    0
## 
## Overall Statistics
##                                           
##                Accuracy : 0.63            
##                  95% CI : (0.5806, 0.6774)
##     No Information Rate : 0.25            
##     P-Value [Acc > NIR] : < 2.2e-16       
##                                           
##                   Kappa : 0.5067          
##                                           
##  Mcnemar's Test P-Value : NA              
## 
## Statistics by Class:
## 
##                      Class: Backdoor Class: OtherType Class: Rootkit
## Sensitivity                   0.9100           0.4700         0.6200
## Specificity                   0.8533           0.9200         0.8800
## Pos Pred Value                0.6741           0.6620         0.6327
## Neg Pred Value                0.9660           0.8389         0.8742
## Prevalence                    0.2500           0.2500         0.2500
## Detection Rate                0.2275           0.1175         0.1550
## Detection Prevalence          0.3375           0.1775         0.2450
## Balanced Accuracy             0.8817           0.6950         0.7500
##                      Class: Spyware Class: Trojan Class: Unknown
## Sensitivity                  0.5200            NA             NA
## Specificity                  0.8533             1              1
## Pos Pred Value               0.5417            NA             NA
## Neg Pred Value               0.8421            NA             NA
## Prevalence                   0.2500             0              0
## Detection Rate               0.1300             0              0
## Detection Prevalence         0.2400             0              0
## Balanced Accuracy            0.6867            NA             NA
##                      Class: Worm
## Sensitivity                   NA
## Specificity                    1
## Pos Pred Value                NA
## Neg Pred Value                NA
## Prevalence                     0
## Detection Rate                 0
## Detection Prevalence           0
## Balanced Accuracy             NA
\end{verbatim}


\end{document}
