\documentclass[]{article}
\usepackage{lmodern}
\usepackage{amssymb,amsmath}
\usepackage{ifxetex,ifluatex}
\usepackage{fixltx2e} % provides \textsubscript
\ifnum 0\ifxetex 1\fi\ifluatex 1\fi=0 % if pdftex
  \usepackage[T1]{fontenc}
  \usepackage[utf8]{inputenc}
\else % if luatex or xelatex
  \ifxetex
    \usepackage{mathspec}
  \else
    \usepackage{fontspec}
  \fi
  \defaultfontfeatures{Ligatures=TeX,Scale=MatchLowercase}
\fi
% use upquote if available, for straight quotes in verbatim environments
\IfFileExists{upquote.sty}{\usepackage{upquote}}{}
% use microtype if available
\IfFileExists{microtype.sty}{%
\usepackage{microtype}
\UseMicrotypeSet[protrusion]{basicmath} % disable protrusion for tt fonts
}{}
\usepackage[margin=1in]{geometry}
\usepackage{hyperref}
\hypersetup{unicode=true,
            pdftitle={Bayesian Poisoning},
            pdfborder={0 0 0},
            breaklinks=true}
\urlstyle{same}  % don't use monospace font for urls
\usepackage{color}
\usepackage{fancyvrb}
\newcommand{\VerbBar}{|}
\newcommand{\VERB}{\Verb[commandchars=\\\{\}]}
\DefineVerbatimEnvironment{Highlighting}{Verbatim}{commandchars=\\\{\}}
% Add ',fontsize=\small' for more characters per line
\usepackage{framed}
\definecolor{shadecolor}{RGB}{248,248,248}
\newenvironment{Shaded}{\begin{snugshade}}{\end{snugshade}}
\newcommand{\AlertTok}[1]{\textcolor[rgb]{0.94,0.16,0.16}{#1}}
\newcommand{\AnnotationTok}[1]{\textcolor[rgb]{0.56,0.35,0.01}{\textbf{\textit{#1}}}}
\newcommand{\AttributeTok}[1]{\textcolor[rgb]{0.77,0.63,0.00}{#1}}
\newcommand{\BaseNTok}[1]{\textcolor[rgb]{0.00,0.00,0.81}{#1}}
\newcommand{\BuiltInTok}[1]{#1}
\newcommand{\CharTok}[1]{\textcolor[rgb]{0.31,0.60,0.02}{#1}}
\newcommand{\CommentTok}[1]{\textcolor[rgb]{0.56,0.35,0.01}{\textit{#1}}}
\newcommand{\CommentVarTok}[1]{\textcolor[rgb]{0.56,0.35,0.01}{\textbf{\textit{#1}}}}
\newcommand{\ConstantTok}[1]{\textcolor[rgb]{0.00,0.00,0.00}{#1}}
\newcommand{\ControlFlowTok}[1]{\textcolor[rgb]{0.13,0.29,0.53}{\textbf{#1}}}
\newcommand{\DataTypeTok}[1]{\textcolor[rgb]{0.13,0.29,0.53}{#1}}
\newcommand{\DecValTok}[1]{\textcolor[rgb]{0.00,0.00,0.81}{#1}}
\newcommand{\DocumentationTok}[1]{\textcolor[rgb]{0.56,0.35,0.01}{\textbf{\textit{#1}}}}
\newcommand{\ErrorTok}[1]{\textcolor[rgb]{0.64,0.00,0.00}{\textbf{#1}}}
\newcommand{\ExtensionTok}[1]{#1}
\newcommand{\FloatTok}[1]{\textcolor[rgb]{0.00,0.00,0.81}{#1}}
\newcommand{\FunctionTok}[1]{\textcolor[rgb]{0.00,0.00,0.00}{#1}}
\newcommand{\ImportTok}[1]{#1}
\newcommand{\InformationTok}[1]{\textcolor[rgb]{0.56,0.35,0.01}{\textbf{\textit{#1}}}}
\newcommand{\KeywordTok}[1]{\textcolor[rgb]{0.13,0.29,0.53}{\textbf{#1}}}
\newcommand{\NormalTok}[1]{#1}
\newcommand{\OperatorTok}[1]{\textcolor[rgb]{0.81,0.36,0.00}{\textbf{#1}}}
\newcommand{\OtherTok}[1]{\textcolor[rgb]{0.56,0.35,0.01}{#1}}
\newcommand{\PreprocessorTok}[1]{\textcolor[rgb]{0.56,0.35,0.01}{\textit{#1}}}
\newcommand{\RegionMarkerTok}[1]{#1}
\newcommand{\SpecialCharTok}[1]{\textcolor[rgb]{0.00,0.00,0.00}{#1}}
\newcommand{\SpecialStringTok}[1]{\textcolor[rgb]{0.31,0.60,0.02}{#1}}
\newcommand{\StringTok}[1]{\textcolor[rgb]{0.31,0.60,0.02}{#1}}
\newcommand{\VariableTok}[1]{\textcolor[rgb]{0.00,0.00,0.00}{#1}}
\newcommand{\VerbatimStringTok}[1]{\textcolor[rgb]{0.31,0.60,0.02}{#1}}
\newcommand{\WarningTok}[1]{\textcolor[rgb]{0.56,0.35,0.01}{\textbf{\textit{#1}}}}
\usepackage{graphicx,grffile}
\makeatletter
\def\maxwidth{\ifdim\Gin@nat@width>\linewidth\linewidth\else\Gin@nat@width\fi}
\def\maxheight{\ifdim\Gin@nat@height>\textheight\textheight\else\Gin@nat@height\fi}
\makeatother
% Scale images if necessary, so that they will not overflow the page
% margins by default, and it is still possible to overwrite the defaults
% using explicit options in \includegraphics[width, height, ...]{}
\setkeys{Gin}{width=\maxwidth,height=\maxheight,keepaspectratio}
\IfFileExists{parskip.sty}{%
\usepackage{parskip}
}{% else
\setlength{\parindent}{0pt}
\setlength{\parskip}{6pt plus 2pt minus 1pt}
}
\setlength{\emergencystretch}{3em}  % prevent overfull lines
\providecommand{\tightlist}{%
  \setlength{\itemsep}{0pt}\setlength{\parskip}{0pt}}
\setcounter{secnumdepth}{0}
% Redefines (sub)paragraphs to behave more like sections
\ifx\paragraph\undefined\else
\let\oldparagraph\paragraph
\renewcommand{\paragraph}[1]{\oldparagraph{#1}\mbox{}}
\fi
\ifx\subparagraph\undefined\else
\let\oldsubparagraph\subparagraph
\renewcommand{\subparagraph}[1]{\oldsubparagraph{#1}\mbox{}}
\fi

%%% Use protect on footnotes to avoid problems with footnotes in titles
\let\rmarkdownfootnote\footnote%
\def\footnote{\protect\rmarkdownfootnote}

%%% Change title format to be more compact
\usepackage{titling}

% Create subtitle command for use in maketitle
\providecommand{\subtitle}[1]{
  \posttitle{
    \begin{center}\large#1\end{center}
    }
}

\setlength{\droptitle}{-2em}

  \title{Bayesian Poisoning}
    \pretitle{\vspace{\droptitle}\centering\huge}
  \posttitle{\par}
    \author{}
    \preauthor{}\postauthor{}
    \date{}
    \predate{}\postdate{}
  

\begin{document}
\maketitle

The purpose of this code snippet is to demonstrate to the reader how a
Naive Bayes (NB) classifier is vulnerable to Bayesian poisoning attacks.
Bayesian poisoning is a technique used by email spammers to compromise
the effectiveness of Bayesian-based spam filters by adding non-spamming
(ham) like words at the end of a spam message. The aim of spammers is to
hamper the probability calculation associated with the NB algorithm. In
other words, this increases the type II probability. In hypothesis
testing, type II error is referred to as ``false negative'', which means
not rejecting the null hypothesis when it is false. Therefore, as a
result of the Bayesian poisoning, the spammers make the spam filtering
system to consider a non-legitimate message a legitimate one. For the
implementation of this code, we use the R language.

\textbf{The dataset:} We utilise the dataset at
\url{https://www.kaggle.com/venky73/spam-mails-dataset}, which contains
a collection of labeled spam and non-spam email messages for this
purpose.

\textbf{Machine learning (ML) task:} Our ML task in this problem would
be to identify the characteristics of spam emails and non-spam emails
accurately. Thereofore, the aim of this is to seperate a spam email from
a non-spam one. This can be considered as a classification problem.
First, we develop a Bayesian-based spam filter using the above email
repository (raw emails) and then perform a Bayesian poisoning attack to
bypass the security control. To this end, we poison the test data into
the spam filter and bypass the security check.

\begin{Shaded}
\begin{Highlighting}[]
\KeywordTok{set.seed}\NormalTok{(}\DecValTok{12345}\NormalTok{) }\CommentTok{# Set the seed value so that the same result is achieved in each replication}

\CommentTok{#install.packages('tm') # Install the required libraries if your system does not have them}
\CommentTok{#install.packages('wordcloud')}
\CommentTok{#install.packages('e1071')}
\CommentTok{#install.packages('gmodels')}
\CommentTok{#install.packages('SnowballC')}

\KeywordTok{library}\NormalTok{(tm) }\CommentTok{# Load the necessary libraries}
\KeywordTok{library}\NormalTok{(wordcloud)}
\KeywordTok{library}\NormalTok{(e1071)}
\KeywordTok{library}\NormalTok{(gmodels)}
\KeywordTok{library}\NormalTok{(SnowballC)}
\KeywordTok{library}\NormalTok{(caret)}
\end{Highlighting}
\end{Shaded}

Load the dataset to the R environment, and remove the term ``Subject:''
as it's common in each message. There is no point of keeping common
constant terms like ``Subject:'' as they don't have discrimination
power.

\begin{Shaded}
\begin{Highlighting}[]
\NormalTok{myData <-}\StringTok{ }\KeywordTok{read.csv}\NormalTok{(}\StringTok{"spam_ham_dataset.csv"}\NormalTok{) }\CommentTok{# read the dataset}
\NormalTok{myData<-myData[,}\KeywordTok{c}\NormalTok{(}\StringTok{"label"}\NormalTok{,}\StringTok{"text"}\NormalTok{)] }\CommentTok{# select two columns}
\NormalTok{myData}\OperatorTok{$}\NormalTok{text<-}\KeywordTok{substring}\NormalTok{(myData}\OperatorTok{$}\NormalTok{text,}\DecValTok{9}\NormalTok{) }\CommentTok{# removing "Subject:"}
\end{Highlighting}
\end{Shaded}

Let's check the word distributions (a.k.a. document terms) in each
class. To this end we use word clouds, in which the size of the word is
proportional to its frequency in the text. If the word clouds are
different, we can conclude that words that appear frequently in spams
differ from words that appear in hams. Therefore, we can use words
(features) in a text message to classify them as ham or spam.

\begin{Shaded}
\begin{Highlighting}[]
\NormalTok{spamMsg<-}\KeywordTok{subset}\NormalTok{(myData,label}\OperatorTok{==}\StringTok{"spam"}\NormalTok{)}
\NormalTok{hamMsg<-}\KeywordTok{subset}\NormalTok{(myData,label}\OperatorTok{==}\StringTok{"ham"}\NormalTok{)}
\CommentTok{# wordcloud(spamMsg$text,scale=c(4,.5),min.freq=5) # plot if the word apperas more than 5 times in the spam messages, makesure to uncomment this line to produce word cloud}
\CommentTok{# wordcloud(hamMsg$text,scale=c(4,.5),min.freq=5) # plot if the word apperas more than 5 times in the ham messages,  makesure to uncomment this line to produce word cloud}
\end{Highlighting}
\end{Shaded}

As we can see in the word cloud outputs, the term distributions differ
between spam and ham, so we can train an ML model (in our case, NB) by
using words as features.

\textbf{Building an NB based spam classifier:} Now we will train our NB
classifier using the above dataset. We will utilize e1071 package in R
for this purpose. To this end, we need to follow following steps,

\textbf{Step 1:} We need to create a document term matrix (DTM) - a
matrix that describes the frequency of terms occured in each message in
our collection.

\begin{Shaded}
\begin{Highlighting}[]
\NormalTok{myCorpus <-}\StringTok{ }\KeywordTok{VCorpus}\NormalTok{(}\KeywordTok{VectorSource}\NormalTok{(myData}\OperatorTok{$}\NormalTok{text)) }\CommentTok{# create a corpus}

\NormalTok{myDTM <-}\StringTok{ }\KeywordTok{DocumentTermMatrix}\NormalTok{(myCorpus, }\DataTypeTok{control =} \KeywordTok{list}\NormalTok{(}
  \DataTypeTok{tolower=}\NormalTok{T,}
  \DataTypeTok{removeNumbers=}\NormalTok{T,}
  \DataTypeTok{removePunctuation=}\NormalTok{T,}
  \DataTypeTok{stopwords =}\NormalTok{ T,}
  \DataTypeTok{stem=}\NormalTok{T}
\NormalTok{))}
\end{Highlighting}
\end{Shaded}

DTM is usually a sparse matrix as most of its entries are filled with
zeros. Let's use only terms which have frequency \textgreater{}=5 as
features in our model building. To this end we use the findFreqTerms ()
function in R as follows. So, we can reduce the number of columns in our
DTM matrix to 8615, which would be a manageable size in our model
building.

\begin{Shaded}
\begin{Highlighting}[]
\NormalTok{freqWords <-}\StringTok{ }\KeywordTok{findFreqTerms}\NormalTok{(myDTM,}\DecValTok{5}\NormalTok{)}
\NormalTok{myDTM <-}\StringTok{ }\NormalTok{myDTM[,freqWords]}
\end{Highlighting}
\end{Shaded}

\textbf{Step 2:} Split the dataset into train and test sets. Later we
are going to poison the test set! We're going to use an 80/20
partitioning for the training and testing. We'll use the
createDataPartition function from the caret package for this purpose.

\begin{Shaded}
\begin{Highlighting}[]
\NormalTok{tr_index <-}\StringTok{ }\KeywordTok{createDataPartition}\NormalTok{(myData}\OperatorTok{$}\NormalTok{label, }\DataTypeTok{p=}\FloatTok{0.80}\NormalTok{, }\DataTypeTok{list=}\OtherTok{FALSE}\NormalTok{) }\CommentTok{# List of 80% of the rows}
\NormalTok{trainSet <-}\StringTok{ }\NormalTok{myData[tr_index,] }\CommentTok{# select 80% of the data for the trainSet}
\NormalTok{testSet <-}\StringTok{ }\NormalTok{myData[}\OperatorTok{-}\NormalTok{tr_index,] }\CommentTok{# Select the remaining 20% of data for testSet}
\end{Highlighting}
\end{Shaded}

Since we are creating our NB model with the DTM entries, we should
obtain corresponding DTM entries for the data in trainSet and testSet.

\begin{Shaded}
\begin{Highlighting}[]
\NormalTok{myDTMTrain <-}\StringTok{ }\NormalTok{myDTM[tr_index,]}
\NormalTok{myDTMTest <-}\StringTok{ }\NormalTok{myDTM[}\OperatorTok{-}\NormalTok{tr_index,]}
\end{Highlighting}
\end{Shaded}

Since we want to represent the presence or absence of a certain word
(feature) in a particular message, we code our DTM as follows.

\begin{Shaded}
\begin{Highlighting}[]
\NormalTok{convert_counts <-}\StringTok{ }\ControlFlowTok{function}\NormalTok{(x) \{}
\NormalTok{  x <-}\StringTok{ }\KeywordTok{ifelse}\NormalTok{(x }\OperatorTok{>}\StringTok{ }\DecValTok{0}\NormalTok{, }\StringTok{"T"}\NormalTok{, }\StringTok{"F"}\NormalTok{)}
\NormalTok{\}}
\NormalTok{myDTMTrainNew <-}\StringTok{ }\KeywordTok{apply}\NormalTok{(myDTMTrain, }\DataTypeTok{MARGIN =} \DecValTok{2}\NormalTok{,convert_counts)}
\NormalTok{myDTMTestNew <-}\StringTok{ }\KeywordTok{apply}\NormalTok{(myDTMTest, }\DataTypeTok{MARGIN =} \DecValTok{2}\NormalTok{, convert_counts)}
\end{Highlighting}
\end{Shaded}

\textbf{Step 3:} Building the NB based spam filter.

\begin{Shaded}
\begin{Highlighting}[]
\NormalTok{NBbasedSpamFilter <-}\StringTok{ }\KeywordTok{naiveBayes}\NormalTok{(myDTMTrainNew, trainSet}\OperatorTok{$}\NormalTok{label) }\CommentTok{# train the model}
\NormalTok{testPredictMsgLabel <-}\StringTok{ }\KeywordTok{predict}\NormalTok{(NBbasedSpamFilter, myDTMTestNew) }\CommentTok{# predict labels for test cases}
\end{Highlighting}
\end{Shaded}

Create the confusion matrix, a table that is often used to describe the
performance of a classifier.

\begin{Shaded}
\begin{Highlighting}[]
\KeywordTok{confusionMatrix}\NormalTok{(testPredictMsgLabel, testSet}\OperatorTok{$}\NormalTok{label, }\DataTypeTok{positive =} \StringTok{"spam"}\NormalTok{) }\CommentTok{# Print confusion matrix}
\end{Highlighting}
\end{Shaded}

\begin{verbatim}
## Confusion Matrix and Statistics
## 
##           Reference
## Prediction ham spam
##       ham  704   67
##       spam  30  232
##                                           
##                Accuracy : 0.9061          
##                  95% CI : (0.8867, 0.9232)
##     No Information Rate : 0.7106          
##     P-Value [Acc > NIR] : < 2.2e-16       
##                                           
##                   Kappa : 0.763           
##                                           
##  Mcnemar's Test P-Value : 0.0002569       
##                                           
##             Sensitivity : 0.7759          
##             Specificity : 0.9591          
##          Pos Pred Value : 0.8855          
##          Neg Pred Value : 0.9131          
##              Prevalence : 0.2894          
##          Detection Rate : 0.2246          
##    Detection Prevalence : 0.2536          
##       Balanced Accuracy : 0.8675          
##                                           
##        'Positive' Class : spam            
## 
\end{verbatim}

\textbf{Bayesian poisoning:} Let's poison the test data to bypass the
above spam filter. To this end, randomly selected non-spam messages are
ammended at the end of spam messages in our test dataset.

\begin{Shaded}
\begin{Highlighting}[]
\CommentTok{### Poison the test data}
\NormalTok{spamTestCases <-}\StringTok{ }\KeywordTok{subset}\NormalTok{(testSet,label}\OperatorTok{==}\StringTok{"spam"}\NormalTok{) }\CommentTok{# select spams in test cases}
\NormalTok{hamTestCases <-}\StringTok{ }\KeywordTok{subset}\NormalTok{(testSet, label}\OperatorTok{==}\StringTok{"ham"}\NormalTok{) }\CommentTok{# select ham in test cases}
\NormalTok{hamMsgInmyData<-}\KeywordTok{subset}\NormalTok{(myData, label}\OperatorTok{==}\StringTok{"ham"}\NormalTok{) }\CommentTok{# select ham in our original dataset}
\NormalTok{hamMsg2ammend<-hamMsgInmyData[}\KeywordTok{sample}\NormalTok{(}\KeywordTok{nrow}\NormalTok{(hamMsgInmyData), }\KeywordTok{nrow}\NormalTok{(spamTestCases)), ] }\CommentTok{# select number of ham messages equal to the spam messages in test cases}
\NormalTok{spamTestCases}\OperatorTok{$}\NormalTok{text <-}\KeywordTok{paste}\NormalTok{(spamTestCases}\OperatorTok{$}\NormalTok{text,hamMsg2ammend}\OperatorTok{$}\NormalTok{text,}\DataTypeTok{sep =} \StringTok{" "}\NormalTok{) }\CommentTok{# ammend selected ham at the end of spams in test cases}
\NormalTok{poisTestData<-}\KeywordTok{rbind}\NormalTok{(hamTestCases,spamTestCases) }\CommentTok{# create the poisoned test set}
\NormalTok{myData[}\OperatorTok{-}\NormalTok{tr_index,]<-poisTestData }\CommentTok{# replace the test entries in original dataset with poisoned test cases}
\end{Highlighting}
\end{Shaded}

Create the new DTM with poisoned data and select high frequency words.
Note that we have to create the DTM and split it as we did with original
data in the above.

\begin{Shaded}
\begin{Highlighting}[]
\NormalTok{myCorpusPoisoned <-}\StringTok{ }\KeywordTok{VCorpus}\NormalTok{(}\KeywordTok{VectorSource}\NormalTok{(myData}\OperatorTok{$}\NormalTok{text))}

\NormalTok{myDTMPoisoned <-}\StringTok{ }\KeywordTok{DocumentTermMatrix}\NormalTok{(myCorpusPoisoned, }\DataTypeTok{control =} \KeywordTok{list}\NormalTok{(}
  \DataTypeTok{tolower=}\NormalTok{T,}
  \DataTypeTok{removeNumbers=}\NormalTok{T,}
  \DataTypeTok{removePunctuation=}\NormalTok{T,}
  \DataTypeTok{stopwords =}\NormalTok{ T,}
  \DataTypeTok{stem=}\NormalTok{T}
\NormalTok{))}

\NormalTok{freqWords <-}\StringTok{ }\KeywordTok{findFreqTerms}\NormalTok{(myDTMPoisoned,}\DecValTok{5}\NormalTok{)}
\NormalTok{myDTMPoisoned <-}\StringTok{ }\NormalTok{myDTMPoisoned[,freqWords]}

\NormalTok{myDTMTestPoisoned <-}\StringTok{ }\NormalTok{myDTMPoisoned[}\OperatorTok{-}\NormalTok{tr_index,] }\CommentTok{# new test set, note that we don't need  a training set as we use the same model created above}

\NormalTok{myDTMTestNewPoisoned <-}\StringTok{ }\KeywordTok{apply}\NormalTok{(myDTMTestPoisoned, }\DataTypeTok{MARGIN =} \DecValTok{2}\NormalTok{, convert_counts) }\CommentTok{# Using the same convert_counts function, we code our poisoned test DTM as did in Step 2 above.}
\end{Highlighting}
\end{Shaded}

Predict labels for poisoned test cases, note that we use the same
NBbasedSpamFilter trained above.

\begin{Shaded}
\begin{Highlighting}[]
\NormalTok{poisonedTestPredictMsgLabel <-}\StringTok{ }\KeywordTok{predict}\NormalTok{(NBbasedSpamFilter, myDTMTestNewPoisoned)}
\end{Highlighting}
\end{Shaded}

\begin{Shaded}
\begin{Highlighting}[]
\KeywordTok{confusionMatrix}\NormalTok{(poisonedTestPredictMsgLabel, testSet}\OperatorTok{$}\NormalTok{label, }\DataTypeTok{positive =} \StringTok{"spam"}\NormalTok{) }\CommentTok{# Print confusion matrix for poisoned data}
\end{Highlighting}
\end{Shaded}

\begin{verbatim}
## Confusion Matrix and Statistics
## 
##           Reference
## Prediction ham spam
##       ham  641  256
##       spam  93   43
##                                          
##                Accuracy : 0.6621         
##                  95% CI : (0.6324, 0.691)
##     No Information Rate : 0.7106         
##     P-Value [Acc > NIR] : 0.9997         
##                                          
##                   Kappa : 0.0204         
##                                          
##  Mcnemar's Test P-Value : <2e-16         
##                                          
##             Sensitivity : 0.14381        
##             Specificity : 0.87330        
##          Pos Pred Value : 0.31618        
##          Neg Pred Value : 0.71460        
##              Prevalence : 0.28945        
##          Detection Rate : 0.04163        
##    Detection Prevalence : 0.13166        
##       Balanced Accuracy : 0.50855        
##                                          
##        'Positive' Class : spam           
## 
\end{verbatim}

As you can see in the confusion matrix, the misclassification of spam as
ham (false negative) increases from 67 to 256. By adding more ham-like
words to the end of spam messages, the attacker can further increase the
false-negative number. As you see in this example, identifying Bayesian
poisoning is imperative. The impact of the misclassification depends on
the domain of the dataset. Especially, applications in medicine and
security, the costs of false negatives would be very high for the
patients or businesses. Therefore, special care should be given to the
diagnosis of Bayesian poisoning.


\end{document}
